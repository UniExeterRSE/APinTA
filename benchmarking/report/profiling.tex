\documentclass[a4paper,titlepage]{article}
\usepackage{hyperref}

\usepackage{fancyhdr}
\usepackage{geometry}
\geometry{left=3.0truecm,right=3.0truecm,top=3.0truecm,bottom=3.0truecm} 

\author{David Acreman}
\title{Profiling Firedrake}

\begin{document}
\pagestyle{fancy}
\lhead{}
\chead{}
\rhead{Profiling report}

\maketitle
\pagebreak
\tableofcontents
\pagebreak

\section{About this document}

This document will become the profiling report deliverable. Profiling is deliverable 5.1.1 and builds are deliverable 5.1.2. \\

\textbf{Q: What is the scope: are we only considering Firedrake? We could separate the building and profiling parts if the profiling needs to be used for other applications.}

\section{Target platforms}

The target HPC platforms for the project are Archer-2 (national tier-1), Isambard (national tier-2) and Isca (local tier-3). Isambard has two separate systems based on the ARM64 architecture: the XCI system which is an established production system, and the A64FX system which is a newer system. The A64FX system should be considered a stretch objective as the hardware and software are less well known to us. We have also purchased a local server which is designed to be similar to an Archer compute node (although it will run Ubuntu which will make it easier to build Firedrake). An overview of the hardware in the target platforms is shown in Table~\ref{tab:hardware}.
%
\begin{table}[htp]
\begin{center}
\begin{tabular}{|l|l|r|r|l|}
\hline 
System         & Processor        & Cores/node & Memory/node     & Interconnect \\
\hline
Archer-2       & AMD x86\_64         & 128        & 256 GB DRAM  & HPE Cray Slingshot  \\
Isambard XCI   & Thunder X2 ARM64    &  64        & 256 GB DRAM  & Cray Aries          \\
Isambard A64FX & Fujitsu A64FX ARM64 & 48         & 32 GB HBM    & Mellanox Infiniband \\
Isca           & Intel x86\_64       & 16/20      & 128 GB DRAM  & Mellanox Infiniband \\
Server         & AMD x86\_64         & 128        & 256 GB DRAM  & None                \\
\hline
\end{tabular}
\end{center}
\caption{Target platform hardware specifications. The Isambard A64FX platform has high bandwidth memory (HBM) which should give better performance than DRAM but has a smaller capacity.}
\label{tab:hardware}
\end{table}%

An overview of the software stacks on the target platforms is shown in Table~\ref{tab:software}. 
The common factor in the compilers is the GNU compiler which is available on all platforms. Each HPC platform has a compiler from the processor vendor (AMD, Intel, ARM and Fujitsu) and Cray systems also have the Cray compiler.
\begin{table}[htp]
\begin{center}
\begin{tabular}{|l|l|l|l|l|}
\hline 
System         & Compilers               & MPI libraries  & Maths libraries & Profilers            \\
\hline
Archer-2       & Cray, GNU, AOCC         & Cray MPICH2     & Cray libsci     & Cray PAT             \\
Isambard XCI   & Cray, GNU, ARM          & Cray MPICH2     & Cray libsci     & Cray PAT, ARM Forge  \\
Isambard A64FX & Cray, GNU, ARM, Fujitsu & Cray MVAPICH2   & Cray libsci     & ARM Forge            \\
Isca           & GNU                     & OpenMPI         & OpenBLAS        & None                  \\
Isca           & Intel                   & Intel           & Intel MKL       & Intel Parallel Studio \\
Server         & GNU                     & MPICH?          & ???             & None                  \\
\hline
\end{tabular}
\end{center}
\caption{Software stacks on target platforms. AOCC is the AMD Optimizing Compiler Collection. Intel MPI and MVAPICH are MPICH derivatives which support an Infiniband interconnect.
\textbf{Note: I can't find a Cray PAT module on the A64FX- should there be one?}}
\label{tab:software}
\end{table}
On the Isambard A64FX system some modules are only available after running 
\begin{verbatim}
module use /lustre/software/aarch64/modulefiles
\end{verbatim}

On the Cray systems compilation is handled by wrapper scripts from the Cray Programming Environment. The wrapper script can run different compilers depending on which programming environment module is loaded at compile time (e.g. \texttt{PrgEnv-cray} calls the Cray compiler and \texttt{PrgEnv-gnu} calls the GNU compiler). The MPI library and maths library are then linked by the wrapper script. On Cray systems the standard profiling tool is Cray Performance Analysis Tools (PAT). On ARM-based Cray systems there is also the ARM/Allinea Forge profiler.

The software environment on Isca is managed using Esaybuild which has the concept of a toolchain. There are two toolchains on Isca: the GCC-foss toolchain and the Intel toolchain. Each toolchain has a different MPI library and maths library. Intel Parallel Studio has an MPI profiling tool called Intel Trace Analyzer and Collector (ITAC) which needs to use Intel MPI.



\section{Builds}

Plan:
\begin{enumerate}
\item Working builds on main target platforms
\item Optimised builds on main target platforms: use optimised maths libraries but beware threading
\end{enumerate}
%
Status:
\begin{itemize}
\item Archer2:
\item Isambard XCI: my build is currently failing due to VTK not building.
\item Server:
\item Isca:
\item Isambard A64FX
\end{itemize}


\section{Profiling}

Within Firedrake and available on all platforms:
\begin{itemize}
\item PyOP2 timed stage
\item PETSc profiling
\end{itemize}
External profilers:
\begin{itemize}
\item ARM/Allinea Forge
\item Cray PAT
\item Intel Parallel Studio
\end{itemize}
%
Plan for testing external profilers:
\begin{enumerate}
\item Test with a simple example e.g. Mandelbrot
\item Test with Python
\item Test with Firedrake
\end{enumerate}

\textbf{What to test? Could look at whether the profiler can spot load imbalance vs. overhead and identify the MPI call responsible.}

\subsection{Cray PAT}

Although this is a Cray tool it does work with compilers other than Cray (the other PrgEnv modules load perftools-base). Experience building SWIFT showed that the \texttt{perftools-lite} module can confused autogen and configure so only load the module prior to the make command. 

\begin{itemize}
\item Cray PAT has two modes: standard and ``lite''
\item There is a graphical viewer (Apprentice 2) which reads the profiling output
\item Can view a time line from a full trace

\subsection{ARM Forge}

\subsection{Intel Parallel Studio}

\end{itemize}

\end{document}
